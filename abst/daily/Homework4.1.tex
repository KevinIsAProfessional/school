% --------------------------------------------------------------
% This is all preamble stuff that you don't have to worry about.
% Head down to where it says "Start here"
% --------------------------------------------------------------
 
\documentclass[12pt]{article}
 
\usepackage[margin=1in]{geometry}
\usepackage{amsmath,amsthm,amssymb}
 
\newcommand{\N}{\mathbb{N}}
\newcommand{\Z}{\mathbb{Z}}
 
\newenvironment{theorem}[2][Theorem]{\begin{trivlist}
\item[\hskip \labelsep {\bfseries #1}\hskip \labelsep {\bfseries #2.}]}{\end{trivlist}}
\newenvironment{lemma}[2][Lemma]{\begin{trivlist}
\item[\hskip \labelsep {\bfseries #1}\hskip \labelsep {\bfseries #2.}]}{\end{trivlist}}
\newenvironment{exercise}[2][Exercise]{\begin{trivlist}
\item[\hskip \labelsep {\bfseries #1}\hskip \labelsep {\bfseries #2.}]}{\end{trivlist}}
\newenvironment{problem}[2][Problem]{\begin{trivlist}
\item[\hskip \labelsep {\bfseries #1}\hskip \labelsep {\bfseries #2.}]}{\end{trivlist}}
\newenvironment{question}[2][Question]{\begin{trivlist}
\item[\hskip \labelsep {\bfseries #1}\hskip \labelsep {\bfseries #2.}]}{\end{trivlist}}
\newenvironment{corollary}[2][Corollary]{\begin{trivlist}
\item[\hskip \labelsep {\bfseries #1}\hskip \labelsep {\bfseries #2.}]}{\end{trivlist}}
\newenvironment{fact}[2][Fact]{\begin{trivlist}
\item[\hskip \labelsep {\bfseries #1}\hskip \labelsep {\bfseries #2.}]}{\end{trivlist}}
\newenvironment{definition}[2][Definition]{\begin{trivlist}
\item[\hskip \labelsep {\bfseries #1}\hskip \labelsep {\bfseries #2.}]}{\end{trivlist}}

\newenvironment{solution}{\begin{proof}[Solution]}{\end{proof}}
 
\begin{document}
 
% --------------------------------------------------------------
%                         Start here
% --------------------------------------------------------------
 
\title{Daily Homework 2-1-21}%replace X with the appropriate number
\author{Kevin Christensen, Joe Kelly\\ %replace with your name
Introduction to Abstract Math} %if necessary, replace with your course title
 
\maketitle
 
\begin{definition}{3.18}
The converse of $A \implies B$ is $B \implies A$.
\end{definition}

\begin{exercise}{3.19} 
Provide an example of a true conditional proposition whose converse is false.
\end{exercise}
 
\begin{proof}
The statement "if $x=5$, then x is odd" is a true conditional statement. However, the converse of the statement is "if x is odd, then $x=5$," and this statement is not true.
\end{proof}
 
 \begin{definition}{3.18}
The inverse of $A \implies B$ is $\neg A \implies \neg B$.
\end{definition}

\begin{exercise}{3.21} 
Provide an example of a true conditional proposition whose inverse is false. 
\end{exercise}
 
\begin{proof}
Let A be "$x=5$" and B be "x is odd". The inverse of the statement $A \implies B$ is "if $x \neq 5$, then x is not odd". This is a false statement because x could be an odd number that is not 5. 
\end{proof}

\begin{exercise}{3.23} 
Let A represent "6 is an even number" and B represent "6 is a multiple of 4". Express the following in ordinary English sentences:
\begin{enumerate}
    \item The converse of $A \implies B$. 
    \item The contrapositive of $A \implies B$. 
\end{enumerate}
\end{exercise}
 
\begin{proof}
\begin{enumerate}
    \item If "6 is a multiple of 4" then "6 is an even number".
    \item If "6 is not a multiple of 4" then "6 is not an even number".
\end{enumerate}
\end{proof}

\begin{exercise}{3.24} 
Find the converse and the contrapositive of the following statement: "If a person lives in Flagstaff then that person lives in Arizona."
\end{exercise}
 
\begin{proof}
\begin{enumerate}
    \item Converse: If a person lives in Arizona, then that person lives in Flagstaff.
    \item Contrapositive: If a person doesn't live in Arizona, then that person does not live in Flagstaff.
\end{enumerate}
\end{proof}
 
\begin{theorem}{3.25}
The implication $A \implies B$ is equivalent to its contrapositive.
\end{theorem}
 
\begin{proof}
\begin{displaymath}
\begin{array}{|c c|c|}
A & B & A \implies B\\
\hline
T & T & T\\
T & F & F\\
F & T & T\\
F & F & T\\
\end{array}
\quad
\begin{array}{|c c|c|}
\neg B & \neg A & \neg B \implies \neg A\\
\hline
T & T & T\\
T & F & F\\
F & T & T\\
F & F & T\\
\end{array}
\end{displaymath}
We see that the truth tables for $A \implies B$ and $\neg B \implies \neg A$ are the same. Therefore, by Definition 3.14, the two statements are logically equivalent.
\end{proof}
 
% --------------------------------------------------------------
%     You don't have to mess with anything below this line.
% --------------------------------------------------------------
 
\end{document}

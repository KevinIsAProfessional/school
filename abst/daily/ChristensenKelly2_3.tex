% --------------------------------------------------------------
% This is all preamble stuff that you don't have to worry about.
% Head down to where it says "Start here"
% --------------------------------------------------------------
 
\documentclass[12pt]{article}
 
\usepackage[margin=1in]{geometry}
\usepackage{amsmath,amsthm,amssymb}
 
\newcommand{\N}{\mathbb{N}}
\newcommand{\Z}{\mathbb{Z}}
 
\newenvironment{theorem}[2][Theorem]{\begin{trivlist}
\item[\hskip \labelsep {\bfseries #1}\hskip \labelsep {\bfseries #2.}]}{\end{trivlist}}
\newenvironment{lemma}[2][Lemma]{\begin{trivlist}
\item[\hskip \labelsep {\bfseries #1}\hskip \labelsep {\bfseries #2.}]}{\end{trivlist}}
\newenvironment{exercise}[2][Exercise]{\begin{trivlist}
\item[\hskip \labelsep {\bfseries #1}\hskip \labelsep {\bfseries #2.}]}{\end{trivlist}}
\newenvironment{problem}[2][Problem]{\begin{trivlist}
\item[\hskip \labelsep {\bfseries #1}\hskip \labelsep {\bfseries #2.}]}{\end{trivlist}}
\newenvironment{question}[2][Question]{\begin{trivlist}
\item[\hskip \labelsep {\bfseries #1}\hskip \labelsep {\bfseries #2.}]}{\end{trivlist}}
\newenvironment{corollary}[2][Corollary]{\begin{trivlist}
\item[\hskip \labelsep {\bfseries #1}\hskip \labelsep {\bfseries #2.}]}{\end{trivlist}}
\newenvironment{fact}[2][Fact]{\begin{trivlist}
\item[\hskip \labelsep {\bfseries #1}\hskip \labelsep {\bfseries #2.}]}{\end{trivlist}}
\newenvironment{definition}[2][Definition]{\begin{trivlist}
\item[\hskip \labelsep {\bfseries #1}\hskip \labelsep {\bfseries #2.}]}{\end{trivlist}}

\newenvironment{solution}{\begin{proof}[Solution]}{\end{proof}}
 
\begin{document}
 
% --------------------------------------------------------------
%                         Start here
% --------------------------------------------------------------
 
\title{Daily Homework 2-3-21}%replace X with the appropriate number
\author{Kevin Christensen, Joe Kelly\\ %replace with your name
Introduction to Abstract Math} %if necessary, replace with your course title
 
\maketitle
 
\begin{problem}{3.27}
Consider the following statement: \\
Assume $n \in \Z$. If $n^2$ is odd, then $n$ is an odd integer.\\
The items below can be assembled to form a proof of this statement, but they are currently out
of order. Put them in the proper order.\\
1. Thus, we assume that n is an even integer.\\
2. We will prove this by contrapositive.\\
3. Since n = 2k, we have that $n^2 = (2k)^2 = 4k^2$.\\
4. Since k is an integer, $2k^2$ is also an integer by Fact 2.3.\\
5. By Definition 2.1, there is an integer k such that n = 2k.\\
6. Since the contrapositive is equivalent to the original statement and we have proved the
contrapositive, the original statement is true.\\
7. By Definition 2.1, $n^2$ is an even integer.\\
8. The contrapositive of the statement “If $n^2$ is odd then n is odd” is “If n is an even integer, then $n^2$ is an even integer.”\\
9. Notice that $n^2 = 2(2k^2)$.
\end{problem}

\begin{solution}

2. We will prove this by contrapositive.\\
8. The contrapositive of the statement “If $n^2$ is odd then n is odd” is “If n is an even integer, then $n^2$ is an even integer.”\\
1. Thus, we assume that n is an even integer.\\
5. By Definition 2.1, there is an integer k such that n = 2k.\\
3. Since n = 2k, we have that $n^2 = (2k)^2 = 4k^2$.\\
9. Notice that $n^2 = 2(2k^2)$.
4. Since k is an integer, $2k^2$ is also an integer by Fact 2.3.\\
7. By Definition 2.1, $n^2$ is an even integer.\\
6. Since the contrapositive is equivalent to the original statement and we have proved the
contrapositive, the original statement is true.\\


Step 2, step 8, step 1, step 5, step 3, step 9, step 4, step 7, step 6.
\end{solution}
\pagebreak
Try proving each of the next three theorems by proving the contrapositive of the given
statement.

\begin{theorem}{3.28} 
Assume $n \in \Z$. If $n^2$ is even, then $n$ is even.
\end{theorem}
 
\begin{proof}
We will prove this by contrapositive. The contrapositive of the statement: ``Assume $n \in \Z$. If $n^2$ is even, then $n$ is even" is ``Assume $n \in \Z$. If $n$ is odd, then $n^2$ is odd." Assume n is odd.  By Problem 2.9, the product of an odd integer and an odd integer is odd. Since $n^2$ is the product of two odd integers, n and n, $n^2$ is an odd integer. Therefore, by proving the contrapositive, by Theorem 3.25, the original statement is true. 


\end{proof}
 
 \begin{theorem}{3.29}
Assume $n,m \in \Z$. If $nm$ is odd, then both $n$ and $m$ are odd.
\end{theorem}

 
\begin{proof}
 We will prove this theorem by contrapositive. The contrapositive of the statement: ``Assume $n,m \in \Z$. If $nm$ is odd, then both $n$ and $m$ are odd." is :``Assume $n,m \in \Z$. If either $n$, $m$ or both are even (by DeMorgan's Law), then $nm$ is even." Let n be even and m be any integer. We proved in Problem 2.10 that if at least one of a pair of integers is even, their product is even. Therefore, nm is even. Thus, by proving the contrapositive, by Theorem 3.25, the original statement is true.
 
 
\end{proof}

\begin{theorem}{3.30} 
Assume $n,m \in \Z$. If $nm$ is even, then either $n$ or $m$ is even.
\end{theorem}

 
\begin{proof}

We will prove this theorem by contrapositive. The contrapositive of the statement ``Assume $n,m \in \Z$. If $nm$ is even, then either $n$ or $m$ is even." is ``Assume $n,m \in \Z$. If both $n$ and $m$ are odd (by DeMorgan's Law), then $nm$ is odd." Let $n$ and $m$ be two odd integers. We proved in problem 2.10 that the product of two odd integers is odd. Therefore, $nm$ is odd. Thus, by proving the contrapositive, by Theorem 3.25, the original statement is true.
 

\end{proof}

% --------------------------------------------------------------
%     You don't have to mess with anything below this line.
% --------------------------------------------------------------
 
\end{document}


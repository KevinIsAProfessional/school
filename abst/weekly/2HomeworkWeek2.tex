% --------------------------------------------------------------
% This is all preamble stuff that you don't have to worry about.
% Head down to where it says "Start here"
% --------------------------------------------------------------
 
\documentclass[12pt]{article}
 
\usepackage[margin=1in]{geometry}
\usepackage{amsmath,amsthm,amssymb}
 
\newcommand{\N}{\mathbb{N}}
\newcommand{\Z}{\mathbb{Z}}
 
\newenvironment{theorem}[2][Theorem]{\begin{trivlist}
\item[\hskip \labelsep {\bfseries #1}\hskip \labelsep {\bfseries #2.}]}{\end{trivlist}}
\newenvironment{lemma}[2][Lemma]{\begin{trivlist}
\item[\hskip \labelsep {\bfseries #1}\hskip \labelsep {\bfseries #2.}]}{\end{trivlist}}
\newenvironment{exercise}[2][Exercise]{\begin{trivlist}
\item[\hskip \labelsep {\bfseries #1}\hskip \labelsep {\bfseries #2.}]}{\end{trivlist}}
\newenvironment{problem}[2][Problem]{\begin{trivlist}
\item[\hskip \labelsep {\bfseries #1}\hskip \labelsep {\bfseries #2.}]}{\end{trivlist}}
\newenvironment{question}[2][Question]{\begin{trivlist}
\item[\hskip \labelsep {\bfseries #1}\hskip \labelsep {\bfseries #2.}]}{\end{trivlist}}
\newenvironment{corollary}[2][Corollary]{\begin{trivlist}
\item[\hskip \labelsep {\bfseries #1}\hskip \labelsep {\bfseries #2.}]}{\end{trivlist}}
\newenvironment{fact}[2][Fact]{\begin{trivlist}
\item[\hskip \labelsep {\bfseries #1}\hskip \labelsep {\bfseries #2.}]}{\end{trivlist}}
\newenvironment{definition}[2][Definition]{\begin{trivlist}
\item[\hskip \labelsep {\bfseries #1}\hskip \labelsep {\bfseries #2.}]}{\end{trivlist}}

\newenvironment{solution}{\begin{proof}[Solution]}{\end{proof}}
 
\begin{document}
 
% --------------------------------------------------------------
%                         Start here
% --------------------------------------------------------------
 
\title{Weekly Homework 2}%replace X with the appropriate number
\author{Kevin Christensen\\ %replace with your name
Introduction to Abstract Math} %if necessary, replace with your course title
 
\maketitle

\begin{definition}{2.1}
An integer is even if n=2k for some integer k
\end{definition}

\begin{definition}{2.2}
An integer is odd if n=2k+1 for some integer k
\end{definition}

\begin{definition}{2.11}
An integer n divides the integer m, written n$\mid$m, if and only if there exists k $\in \Z$ such that m=nk. We may also say that m is divisible by n.
\end{definition}

\begin{fact}{2.3.1} 
Sums and products of integers are integers.
\end{fact}

\begin{problem}{2.10} %You can use theorem, exercise, problem, or question here.  Modify x.yz to be whatever number you are proving
Either prove, or provide a counterexample to, the statement "If at least one of a pair of integers is even, then their product is even."
\end{problem}
 
\begin{proof}
 %You can also use solution in place of proof.
Consider m,n $\in \Z$, such that n is even. By Definition 2.1, we know n=2k.
\begin{align*}
    n*m & = (2k)*m\\
& = 2(km)
\end{align*}
By Fact 2.3.2, we know that $km$ is an integer. Therefore, 2$km$ is even by Definition 2.1.
\end{proof}
 
\begin{theorem}{2.15}
Assume n,m,a $\in \Z$. If a$\mid$n, then a$\mid$nm.
\end{theorem}
 
\begin{proof}
  Assume n,m,a $\in \Z$, and that a$\mid$n. By Definition 2.11, in order for the statement a$\mid$mn to be true, we know that $\frac{mn}{a}$ must be an integer.\\
\begin{align*}
    \frac{mn}{a} & = \frac{m(ak)}{a} & (\text{By Definition 2.11})\\
    & = mk
\end{align*}
By Fact 2.3.1, we know that mk is an integer. Therefore, we know a$\mid$mn is true.
\end{proof}

% --------------------------------------------------------------
%     You don't have to mess with anything below this line.
% --------------------------------------------------------------
 
\end{document}

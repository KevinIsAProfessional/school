% --------------------------------------------------------------
% This is all preamble stuff that you don't have to worry about.
% Head down to where it says "Start here"
% --------------------------------------------------------------
 
\documentclass[12pt]{article}
 
\usepackage[margin=1in]{geometry}
\usepackage{amsmath,amsthm,amssymb}
 
\newcommand{\N}{\mathbb{N}}
\newcommand{\Z}{\mathbb{Z}}
 
\newenvironment{theorem}[2][Theorem]{\begin{trivlist}
\item[\hskip \labelsep {\bfseries #1}\hskip \labelsep {\bfseries #2.}]}{\end{trivlist}}
\newenvironment{lemma}[2][Lemma]{\begin{trivlist}
\item[\hskip \labelsep {\bfseries #1}\hskip \labelsep {\bfseries #2.}]}{\end{trivlist}}
\newenvironment{exercise}[2][Exercise]{\begin{trivlist}
\item[\hskip \labelsep {\bfseries #1}\hskip \labelsep {\bfseries #2.}]}{\end{trivlist}}
\newenvironment{problem}[2][Problem]{\begin{trivlist}
\item[\hskip \labelsep {\bfseries #1}\hskip \labelsep {\bfseries #2.}]}{\end{trivlist}}
\newenvironment{question}[2][Question]{\begin{trivlist}
\item[\hskip \labelsep {\bfseries #1}\hskip \labelsep {\bfseries #2.}]}{\end{trivlist}}
\newenvironment{corollary}[2][Corollary]{\begin{trivlist}
\item[\hskip \labelsep {\bfseries #1}\hskip \labelsep {\bfseries #2.}]}{\end{trivlist}}
\newenvironment{fact}[2][Fact]{\begin{trivlist}
\item[\hskip \labelsep {\bfseries #1}\hskip \labelsep {\bfseries #2.}]}{\end{trivlist}}
\newenvironment{definition}[2][Definition]{\begin{trivlist}
\item[\hskip \labelsep {\bfseries #1}\hskip \labelsep {\bfseries #2.}]}{\end{trivlist}}

\newenvironment{solution}{\begin{proof}[Solution]}{\end{proof}}
 
\begin{document}
 
% --------------------------------------------------------------
%                         Start here
% --------------------------------------------------------------
 
\title{Weekly Homework 1}%replace X with the appropriate number
\author{Kevin Christensen\\ %replace with your name
Introduction to Abstract Math} %if necessary, replace with your course title
 
\maketitle

\begin{definition}{2.1}
An integer is even if n=2k for some integer k
\end{definition}

\begin{definition}{2.2}
An integer is odd if n=2k+1 for some integer k
\end{definition}

\begin{fact}{2.3.1} 
Sums and products of integers are integers.
\end{fact}

\begin{theorem}{2.4} %You can use theorem, exercise, problem, or question here.  Modify x.yz to be whatever number you are proving
If n $\in \Z$ then the sum of n and n+1 is odd.
\end{theorem}

\begin{proof} %You can also use solution in place of proof.
Let n be an integer. Notice that n+(n+1)=2n+1. By Definition 2.2, we see that 2n+1 is odd.
\end{proof}

\begin{problem}{2.6}
Either prove, or provide a counterexample to, the statement "The sum of an even integer and an odd integer is odd."
\end{problem}
 
\begin{proof}
Let n be an even integer and m be an odd integer. By Definitions 2.1 and 2.2 respectively, we can write n as 2k and m as 2j+1.
\begin{align*}
    n+m& = 2k+2j+1\\
    & = 2(k+j)+1\\
\end{align*}
We know that (k+j) is an integer by Fact 2.3.1. Therefore, we show by Definition 2.2 that 2(k+j)+1 is odd for all values of k and j.
\end{proof}
 
% --------------------------------------------------------------
%     You don't have to mess with anything below this line.
% --------------------------------------------------------------
 
\end{document}

% --------------------------------------------------------------
% This is all preamble stuff that you don't have to worry about.
% Head down to where it says "Start here"
% --------------------------------------------------------------
 
\documentclass[12pt]{article}
 
\usepackage[margin=1in]{geometry}
\usepackage{amsmath,amsthm,amssymb}
 
\newcommand{\N}{\mathbb{N}}
\newcommand{\Z}{\mathbb{Z}}
 
\newenvironment{theorem}[2][Theorem]{\begin{trivlist}
\item[\hskip \labelsep {\bfseries #1}\hskip \labelsep {\bfseries #2.}]}{\end{trivlist}}
\newenvironment{lemma}[2][Lemma]{\begin{trivlist}
\item[\hskip \labelsep {\bfseries #1}\hskip \labelsep {\bfseries #2.}]}{\end{trivlist}}
\newenvironment{exercise}[2][Exercise]{\begin{trivlist}
\item[\hskip \labelsep {\bfseries #1}\hskip \labelsep {\bfseries #2.}]}{\end{trivlist}}
\newenvironment{problem}[2][Problem]{\begin{trivlist}
\item[\hskip \labelsep {\bfseries #1}\hskip \labelsep {\bfseries #2.}]}{\end{trivlist}}
\newenvironment{question}[2][Question]{\begin{trivlist}
\item[\hskip \labelsep {\bfseries #1}\hskip \labelsep {\bfseries #2.}]}{\end{trivlist}}
\newenvironment{corollary}[2][Corollary]{\begin{trivlist}
\item[\hskip \labelsep {\bfseries #1}\hskip \labelsep {\bfseries #2.}]}{\end{trivlist}}
\newenvironment{fact}[2][Fact]{\begin{trivlist}
\item[\hskip \labelsep {\bfseries #1}\hskip \labelsep {\bfseries #2.}]}{\end{trivlist}}
\newenvironment{definition}[2][Definition]{\begin{trivlist}
\item[\hskip \labelsep {\bfseries #1}\hskip \labelsep {\bfseries #2.}]}{\end{trivlist}}

\newenvironment{solution}{\begin{proof}[Solution]}{\end{proof}}
 
\begin{document}
 
% --------------------------------------------------------------
%                         Start here
% --------------------------------------------------------------
 
\title{Daily Homework 2.2}%replace X with the appropriate number
\author{Kevin Christensen and Joe Kelly\\ %replace with your name
Introduction to Abstract Math} %if necessary, replace with your course title
 
\maketitle

 
\begin{theorem}{2.15} 
 Assume n,m,a $\in \Z$. If a$\mid$n, then a$\mid$mn.
\end{theorem}
 
\begin{proof}  
By Definition 2.11, we can write mn = m(ak), where k $\in \Z$. 
\begin{align*}
\frac{mn}{a} & = \frac{m(ak)}{a} \\
& = km.
\end{align*}We know by Fact 2.3 that $km$ is an integer. Therefore, by Definition 2.11, a$\mid$mn.
\end{proof}

\begin{corollary}{2.16}
Assume n,a $\in \Z$. If a divides n, then a divides $n^{2}$.
\end{corollary}

\begin{proof}
By Theorem 2.15, we know that if a$\mid$n then a$\mid$mn. If m = n, then mn = $n^2$. Therefore, since a$\mid$n, we know that a$\mid$$n^2$.
\end{proof}

 
\begin{problem}{2.17} 
Assume n,a $\in \Z$. Consider the statement:
\begin{center}
\fbox{If a divides $n^{2}$ then a divides n.}\\
\end{center}
Is this statement always true? Is it always false? Prove that your answers are correct by giving particular examples and/or general arguments.
\end{problem}
 
\begin{proof} 
This statement is only sometimes true.\\
Consider a case described by Corollary 2.16. Here, a$\mid$${n^2}$ and a$\mid$n. This is a case where the statement is true. \\
Now consider the case where a = 8 and n = 4. ${n^2}$/a = 16/8 = 2 $\in \Z$, so a$\mid$${n^2}$. However, n/a = 4/8 = 1/2 $\notin \Z$. In this case, the statement is false because a divides $n^2$ but not n.
\end{proof}

\begin{theorem}{2.18}
Assume n,a $\in \Z$. If a divides n, then a divides -n.
\end{theorem}
 
\begin{proof}  
Let m = -1. By Theorem 2.15, if a$\mid$n, a$\mid$mn. Therefore, if a divides n, a also divides -n. \\

\end{proof}

% --------------------------------------------------------------
%     You don't have to mess with anything below this line.
% --------------------------------------------------------------
 
\end{document}

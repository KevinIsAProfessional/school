% --------------------------------------------------------------
% This is all preamble stuff that you don't have to worry about.
% Head down to where it says "Start here"
% --------------------------------------------------------------
 
\documentclass[12pt]{article}
 
\usepackage[margin=1in]{geometry}
\usepackage{amsmath,amsthm,amssymb}
\usepackage{dsfont}
 
\newcommand{\N}{\mathbb{N}}
\newcommand{\Z}{\mathbb{Z}}
 
\newenvironment{theorem}[2][Theorem]{\begin{trivlist}
\item[\hskip \labelsep {\bfseries #1}\hskip \labelsep {\bfseries #2.}]}{\end{trivlist}}
\newenvironment{lemma}[2][Lemma]{\begin{trivlist}
\item[\hskip \labelsep {\bfseries #1}\hskip \labelsep {\bfseries #2.}]}{\end{trivlist}}
\newenvironment{exercise}[2][Exercise]{\begin{trivlist}
\item[\hskip \labelsep {\bfseries #1}\hskip \labelsep {\bfseries #2.}]}{\end{trivlist}}
\newenvironment{problem}[2][Problem]{\begin{trivlist}
\item[\hskip \labelsep {\bfseries #1}\hskip \labelsep {\bfseries #2.}]}{\end{trivlist}}
\newenvironment{question}[2][Question]{\begin{trivlist}
\item[\hskip \labelsep {\bfseries #1}\hskip \labelsep {\bfseries #2.}]}{\end{trivlist}}
\newenvironment{corollary}[2][Corollary]{\begin{trivlist}
\item[\hskip \labelsep {\bfseries #1}\hskip \labelsep {\bfseries #2.}]}{\end{trivlist}}
\newenvironment{fact}[2][Fact]{\begin{trivlist}
\item[\hskip \labelsep {\bfseries #1}\hskip \labelsep {\bfseries #2.}]}{\end{trivlist}}
\newenvironment{definition}[2][Definition]{\begin{trivlist}
\item[\hskip \labelsep {\bfseries #1}\hskip \labelsep {\bfseries #2.}]}{\end{trivlist}}


\newenvironment{solution}{\begin{proof}[Solution]}{\end{proof}}
 
\begin{document}
 
% --------------------------------------------------------------
%                         Start here
% --------------------------------------------------------------
 
\title{Daily Homework 6.1}%replace X with the appropriate number
\author{Kevin Christensen, Ailey Robinson\\ %replace with your name
Introduction to Abstract Math} %if necessary, replace with your course title
 
\maketitle
 
\begin{theorem}{3.61}
For all integers $3n^2 + n + 14$ is even.
\end{theorem}
 
\begin{proof}
There are two possibilities for $n$, either $n$ is even or $n$ is odd. If $n$ is even, we express it as $n = 2k$ and substitute that into the equation, giving us $3n^2 + n + 14 = 12k^2 + 2k + 14 = 2(6k^2 + k + 7)$. By Definition 2.1, this is an even integer.\\
If $n$ is odd, we can express it as $n = 2k + 1$. This gives us the equation $3n^2 + n + 14 = 12k^2 + 12k + 3 + 2k + 1 + 14 = 2(6k^2 + 7k + 9)$. By Definition 2.1, this is also an even integer.\\
Because both possibilities for $n$ make $3n^2 + n + 14$ even, we know that the statement is true for all integers $n$.
\end{proof}
 
\begin{exercise}{4.4}
Unpack each of the following sets and see if you can find a simple description of the elements that each set contains.
\begin{enumerate}
    \item A = $\{$ $x \in \N \mid x=3k$ for some $k \in N$ $\}$
    \item B = $\{$ $t \in \mathds{R} \mid t^2 \le 2$ $\}$
    \item C = $\{$ $t \in \Z \mid t^2 \le 2$ $\}$
    \item D = $\{$ $m \in \mathds{R} \mid m=1-\frac{1}{n}$, where $n \in N$ $\}$
\end{enumerate}
\end{exercise}


\begin{solution} $\\$
\begin{enumerate}
    \item A is the set of natural numbers that are divisible by 3.
    \item B is the set of rational numbers that are between -$\sqrt{2}$ and $\sqrt{2}$.
    \item C is the set of integers that are between -$\sqrt{2}$ and $\sqrt{2}$.
    \item D is the set of real numbers between 0 and 2.
\end{enumerate}
\end{solution}

\begin{exercise}{4.5}
Write each of the following sentences using set builder notation.
\begin{enumerate}
    \item The set of all real numbers less than -$\sqrt{2}$.
    \item The set of all real numbers greater than -12 and less than or equal to 42.
    \item The set of all even natural numbers.
\end{enumerate}
\end{exercise}

\begin{solution}$ $\\
\begin{enumerate}
    \item A = $\{$ $x \in \mathds{R} \mid x < -\sqrt{2}$ $\}$
    \item B = $\{$ $y \in \mathds{R} \mid -12 < x \le 42$ $\}$
    \item C = $\{$ $z \in \N \mid z = 2k, k \in \Z$ $\}$
\end{enumerate}

\end{solution}


\begin{problem}{4.7}
Suppose $A$ and $B$ are sets. Describe a skeleton proof for proving that $A \subseteq B$.
\end{problem}

 
\begin{solution}
Let $x \in A$.
\begin{center}... [$\textit{Use definitions and known results to prove}$ $x \in B.$] ...\end{center}
By Definition 4.6, $A \subseteq B$.

\end{solution}

\begin{theorem}{4.8}
Let $S$ be a set. Then $S \subseteq S$ and $\emptyset \subseteq S$.
\end{theorem}

\begin{proof}$ $ \\
Assume $x \in S$. For any chosen $x$, we know that $x \in S$. Therefore, by Definition 4.6, $S \subseteq S$.\\
Since $\emptyset$ contains no elements, we can always take the entirety of the empty set from any set containing 0 or more elements. Therefore, $\emptyset \in S$.
\end{proof}
 
% --------------------------------------------------------------
%     You don't have to mess with anything below this line.
% --------------------------------------------------------------
 
\end{document}
